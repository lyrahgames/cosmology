\documentclass{article}
\usepackage{standard}
\usepackage{times}

\geometry{a4paper, left=30mm, right=30mm, top=30mm, bottom=30mm}
\linespread{1.0}

\begin{document}
  \hrule
  \smallskip
  \hrule
  \begin{multicols}{2}
    \tableofcontents
  \end{multicols}
  % \newpage
  \hrule
  \smallskip
  \hrule
  \smallskip

  \begin{multicols}{2}

  \section{Einführung: Der Gegenstand der Kosmologie} % (fold)
  \label{sec:einführung_der_gegenstand_der_kosmologie}
    \begin{itemize}
      \item Wieso ist Kosmologie überhaupt möglich?
      \item Definition: Kosmologie
      \item Definition: Universum
      \item Gedankenexperiment zu Teilchenhorizonten
      \item Kosmologisches Modell
      \item Fragen der Kosmologie
    \end{itemize}
  % section einführung_der_gegenstand_der_kosmologie (end)

  \section{Kinematik von Robertson-Walker-Kosmen} % (fold)
  \label{sec:kinematik_von_robertson_walker_kosmen}
    \subsection{Kosmologische Symmetrien und Prinzipien} % (fold)
    \label{sub:kosmologische_symmetrien_und_prinzipien}
      \begin{itemize}
        \item Definition: Kosmologisches Prinzip
        \item Bemerkung
      \end{itemize}
    % subsection kosmologische_symmetrien_und_prinzipien (end)

    \subsection{Robertson-Walker-Metrik} % (fold)
    \label{sub:robertson_walker_metrik}
      \begin{itemize}
        \item Vorüberlegung: Trennung des räumlichen und zeitlichen Anteils
        \item 1.~Schritt: Lichtausbreitung
        \item 2.~Schritt: Weltzeit und das Weylsche Postulat
        \item 3.~Schritt: Skalenfaktor
        \item 4.~Schritt: Räume konstanter Krümmung
        \item 5.~Schritt: Reskalierung
        \item Definition: Robertson-Walker-Metrik
        \item Definition: χ-Koordinate
        \item Beispiel: Volumen des Universums
        \item Geschwindigkeits-Entfernungs-Zusammenhang
      \end{itemize}
    % subsection robertson_walker_metrik (end)

    \subsection{Die Expansions-Rotverschiebung} % (fold)
    \label{sub:die_expansions_rotverschiebung}
      \begin{itemize}
        \item Herleitung der Expansions-Rotverschiebung
        \item Bemerkung: Unterschied zum Doppler-Effekt
        \item Definition: Rotverschiebung
      \end{itemize}
    % subsection die_expansions_rotverschiebung (end)

    \subsection{Kosmische Nahzone} % (fold)
    \label{sub:kosmische_nahzone}
      \begin{itemize}
        \item Definition: Hubble-Parameter $H$, Beschleunigungs-Parameters $q$
        \item Approximation: Rotverschiebungs-Entfernungs-Zusammenhang
        \item Verschiedene Entfernungen
        \item Herleitung der Leuchtkraft-Entfernung
        \item Definition: Leuchtkraft-Entfernung
        \item Approximation: Leuchtkraft-Entfernung in Kosmischer Nahzone, Hubble-Diagramm
        \item Definition: Winkeldurchmesser-Entfernung
        \item Hubble-Effekt
      \end{itemize}
    % subsection kosmische_nahzone (end)
  % section kinematik_von_robertson_walker_kosmen (end)

  \section{Dynamik von Robertson-Walker-Kosmen} % (fold)
  \label{sec:dynamik_von_robertson_walker_kosmen}
    \subsection{Friedmansche Modelle} % (fold)
    \label{sub:friedmansche_modelle}
      \begin{itemize}
        \item Einsteinschen Feldgleichungen
        \item Bianchi-Identitäten, Kontinuitätsgleichungen
        \item Kosmologische Konstante
        \item Einsetzen der Robertson-Walker-Metrik in den Einstein-Tensor
        \item Allgemeinste Lösung: ideale Flüssigkeiten, Energie-Impuls-Tensor
        \item Definition: Beschleunigungsgleichung, Friedman-Gleichung, Kontinuitätsgleichung
        \item Lösungsstrategie
        \item Zustandsgleichung: Vakuum, Strahlung, inkohärente Materie wie Staub
      \end{itemize}
    % subsection friedmansche_modelle (end)

    \subsection{Friemansche Staubkosmen} % (fold)
    \label{sub:friemansche_staubkosmen}
      \begin{itemize}
        \item Annahmen
        \item Effektives Potential
        \item Lösungen für $ε\in\set{-1,0,1}{}$
        \item Bedeutung
      \end{itemize}
    % subsection friemansche_staubkosmen (end)

    \subsection{Die Kritische Dichte} % (fold)
    \label{sub:die_kritische_dichte}
      \begin{itemize}
        \item Definition: kritische Dichte
        \item Werte der kritischen Dichte
        \item Definition: $\Omega$-Parameter
      \end{itemize}
    % subsection die_kritische_dichte (end)

    \subsection{Das Weltalter} % (fold)
    \label{sub:das_weltalter}
      \begin{itemize}
        \item Definition: Weltalter
        \item Substitution durch Rotverschiebung
        \item Werte des Weltalters
        \item Gleichungen für $ε\in\set{-1,0,1}{}$
        \item Bemerkung: Hubble-Alter versus Hubble-Zeit
      \end{itemize}
    % subsection das_weltalter (end)

    \subsection{Rotverschiebungs-Entfernungs-Relation} % (fold)
    \label{sub:rotverschiebungs_entfernungs_relation}
      \begin{itemize}
        \item Beispielherleitung der Mattig-Formel
        \item Bemerkungen zur Mattig-Formel
      \end{itemize}
    % subsection rotverschiebungs_entfernungs_relation (end)

    \subsection{Friedman-artige Strahlungskosmen} % (fold)
    \label{sub:friedman_artige_strahlungskosmen}
      \begin{itemize}
        \item Annahmen
        \item Lösung
      \end{itemize}
    % subsection friedman_artige_strahlungskosmen (end)

    \subsection{Staub und Strahlung} % (fold)
    \label{sub:staub_und_strahlung}
      \begin{itemize}
        \item Definition: Kosmische Parameter $\Omega_M$, $\Omega_R$, $\Omega_K$
        \item Bemerkung: Diskussion dominierende Parameter
      \end{itemize}
    % subsection staub_und_strahlung (end)

    \subsection{Leistungen und Grenzen der Newtonschen Kosmologie} % (fold)
    \label{sub:leistungen_und_grenzen_der_newtonschen_kosmologie}
      \begin{itemize}
        \item Herleitung der Friedman-Gleichung
        \item Gravitationsparadoxon
      \end{itemize}
    % subsection leistungen_und_grenzen_der_newtonschen_kosmologie (end)
  % section dynamik_von_robertson_walker_kosmen (end)

  \section{Kosmologisch Relevante Astronomische Beobachtungen} % (fold)
  \label{sec:kosmologisch_relevante_astronomische_beobachtungen}
    \subsection{Dunkelheit des Nachthimmels} % (fold)
    \label{sub:dunkelheit_des_nachthimmels}
      \begin{itemize}
        \item Gedankenexperiment: Wald mit Bäumen
        \item Weltmodelle und Erklärungsversuche
        \item Wirkung der endlichen Leuchtdauer
        \item Wirkung der Expansions-Rotverschiebung
        \item Wirkung der Expansion
        \item kosmische Hintergrundstrahlung
      \end{itemize}
    % subsection dunkelheit_des_nachthimmels (end)

    \subsection{Die Kosmische Entfernungsskala} % (fold)
    \label{sub:die_kosmische_entfernungsskala}
      \begin{itemize}
        \item Leuchtkraft-Entfernung
        \item Entfernungsmodul
        \item Cepheiden als Standardkerzen
        \item Supernovae Ia als Standardkerzen
        \item Hubble-Diagramm und die beschleunigte Expansion
      \end{itemize}
    % subsection die_kosmische_entfernungsskala (end)

    \subsection{Die Kosmische Mikrowellen-Hintergrund-Strahlung} % (fold)
    \label{sub:die_kosmische_mikrowellen_hintergrund_strahlung}
      \begin{itemize}
        \item Temperatur und Plancksches Spektrum
        \item Plancksches Spektrum und Expansion
        \item Energiedichte
        \item Anzahldichte der Photonen
      \end{itemize}
    % subsection die_kosmische_mikrowellen_hintergrund_strahlung (end)
  % section kosmologisch_relevante_astronomische_beobachtungen (end)

  \section{Die Kosmologische Konstante} % (fold)
  \label{sec:die_kosmologische_konstante}
    \subsection{Einstein Kosmos} % (fold)
    \label{sub:einstein_kosmos}
      \begin{itemize}
        \item Annahmen
        \item Lösung
      \end{itemize}
    % subsection einstein_kosmos (end)

    \subsection{Friedman-Lemaître-Kosmen} % (fold)
    \label{sub:friedman_lemaître_kosmen}
      \begin{itemize}
        \item Annahmen
        \item Analyse des effektiven Potentials
        \item Lösungsmodelle des Diagramms und qualitative Betrachtung
        \item Friedman-Lemaître-Modelle: Annahmen und Lösungen
      \end{itemize}
    % subsection friedman_lemaître_kosmen (end)

    \subsection{Das Vakuum als ideale Flüssigkeit} % (fold)
    \label{sub:das_vakuum_als_ideale_flüssigkeit}
      \begin{itemize}
        \item Interpretation von $\Lambda$
        \item Dunkle Energie
        \item Phantomenergie
      \end{itemize}
    % subsection das_vakuum_als_ideale_flüssigkeit (end)

    \subsection{Die Kosmologischen Parameter} % (fold)
    \label{sub:die_kosmologischen_parameter}
      \begin{itemize}
        \item Parameter-Gleichung
        \item Beschleunigungsparameter-Gleichung
        \item Rotverschiebung des Wendepunkts
      \end{itemize}
    % subsection die_kosmologischen_parameter (end)

    \subsection{Das Weltalter} % (fold)
    \label{sub:das_weltalter}
      \begin{itemize}
        \item Integrand des Weltalters
        \item Lösung für $ε=0$
      \end{itemize}
    % subsection das_weltalter (end)

    \subsection{Der de Sitter-Kosmos} % (fold)
    \label{sub:der_de_sitter_kosmos}
      \begin{itemize}
        \item Annahmen
        \item Lösungen mit Diagramm
      \end{itemize}
    % subsection der_de_sitter_kosmos (end)
  % section die_kosmologische_konstante (end)

  \section{Horizonte im Universum} % (fold)
  \label{sec:horizonte_im_universum}
    \subsection{Konforme Raumzeit und gerade Lichtkegel} % (fold)
    \label{sub:konforme_raumzeit_und_gerade_lichtkegel}
      \begin{itemize}
        \item Definition: Konformzeit, gerader Lichtkegel
      \end{itemize}
    % subsection konforme_raumzeit_und_gerade_lichtkegel (end)

    \subsection{Teilchenhorizonte und die Hubble-Kugel} % (fold)
    \label{sub:teilchenhorizonte_und_die_hubble_kugel}
      \begin{itemize}
        \item Hubble-Kugel
      \end{itemize}
    % subsection teilchenhorizonte_und_die_hubble_kugel (end)
  % section horizonte_im_universum (end)

  \end{multicols}
\end{document}